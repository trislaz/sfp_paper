\section{Discussion}

In pathology , expert annotations are usually hard to obtain. However, we are often in a situation where a small quantity of labeled annotation exists but not in sufficient quantities to support  fully supervised techniques. Yet, even in small quantities, expert annotations carry meaningful information that one could use to enforce biological context to deep learning models and make sure that networks learn appropriate patterns. On the other hand, self-supervised methods have proven their efficacy to extract generic features in the histopathological domain and their usefulness for downstream supervision tasks, even in the absence of massive ground truth data. Methods capable of reconciling self-supervision with strong supervision can therefore be useful and open the door to better performances. \\

In this paper, we presented a way to inject the fine-grained tile level information by fine-tuning the feature extractor with a joint optimization process. This process allowed to mix self-supervised learning features with tile classification ones and helped the downstream WSI classification task.  \\

We applied our method to the TissueNet Challenge, a challenge for the automatic grading of cervix cancer, that provided annotations at the slide and tile level, thus representing an appropriate use case to validate our method of mixed supervision. We also propose in this study insights and guidelines for the training of a WSI classifier in the presence of tile annotations. \\

First, we showed that SSL is always beneficial to our downstream WSI classification tasks. Fine-tuning pre-trained weights with SSL for only 50 hours brings a 4\% improvement over WSI classification weighted accuracy, and near to 5\% when fine-tuning for longer (100 epochs). \\

Second, a small set of annotated tiles can bring benefit to the WSI classification task, up to 2\% of weighted accuracy for a supervised dataset of around 5000 images. \\
Such a set of tiles can be obtained easily by asking the pathologist to select a few ROIs that guided his decision while labeling the WSIs, which can be achieved without a strong time commitment.
However this boost in performance can be reached only if the feature extractor is pre-trained with SSL, and for sufficiently long: SSL unlocks the supervised fine-tuning benefits.\\

To further understand the differences between the range of supervision used to extract tile features, we conducted qualitative analysis on features visualizations by activation maximization and observed that features obtained from SSL, supervised or mixed trainings were more relevant for histological tasks and that class-discriminative patterns were indeed identified by the model. We also observed that supervised training on the tiles alone led to much less diversity in the features extracted by the model than the ones obtained with SSL.\\

The scope of this study contains by design three limitations.
First, SSL models were trained by fine-tuning already pre-trained weights on imagenet. This may explain the rapid convergence and boost in performance observed; However it may also underestimate this boost if the SSL models were trained from scratch. We did not compare SSL trained from scratch and fine-tuned SSL, and left it to future work. \\

Second, all the conclusions reached are conditioned by the fact that we do not fine-tune the feature extractor network during the WSI classification training. Keeping these weights frozen, and even pre-computing the tiles representations brings a large computational benefit (both in memory and speed of computations), but prevents the feature extractor from specializing during the WSI classification training. \\

Third, the tendency observed in table \ref{tab:annots_epochs} of better performances correlated with larger numbers of annotations is modest and would require more annotations to validate it. \\

Finally, our method can be improved in several ways. First, SimCLR, was a pioneer method in self-supervised learning architecture and has proven to be efficient but it suffers from high performance drop when decreasing the batch size \citep{chen_simple_2020}. Other SSL models have been developed to alleviate this limitation. MoCo \citep{he_momentum_2020} actually propose a momentum mechanism allowing optimal performances even without large batch size and therefore, numerous available parallel GPUs. Other models like VICReg \citep{bardes_vicreg_2021} proposed  techniques to maximize the variance between the features and therefore limit their redundancies. It will be interesting to benchmark these SSL variants with respect to their impact on WSI classification accuracy and feature interpretability. \\

To conclude, we present a method that provides an interesting alternative to using full supervision, pre-training on unrelated data sets or self-supervision. We convincingly show that the learned feature representations are both leading to higher performance and providing more intermediate features that are more adapted to the problem and point to relevant cell and tissue phenotypes. We expect that the mixed supervision will be adopted by the field and lead to better models. 






