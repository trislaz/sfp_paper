%% This is file `medima-template.tex',
%% 
%% Copyright 2018 Elsevier Ltd
%% 
%% This file is part of the 'Elsarticle Bundle'.
%% ---------------------------------------------
%% 

%% It may be distributed under the conditions of the LaTeX Project Public
%% License, either version 1.2 of this license or (at your option) any
%% later version.  The latest version of this license is in
%%    http://www.latex-project.org/lppl.txt
%% and version 1.2 or later is part of all distributions of LaTeX
%% version 1999/12/01 or later.
%% 
%% The list of all files belonging to the 'Elsarticle Bundle' is
%% given in the file `manifest.txt'.
%% 
%% Template article for Elsevier's document class `elsarticle'
%% with harvard style bibliographic references
%%
%% $Id: medima-template.tex 153 2018-12-01 11:38:32Z rishi $
%% $URL: http://lenova.river-valley.com/svn/elsarticle/trunk/medima-template.tex $
%%
%% Use the option review to obtain double line spacing
%\documentclass[times,review,preprint,authoryear]{elsarticle}

%% Use the options `twocolumn,final' to obtain the final layout
%% Use longtitle option to break abstract to multiple pages if overfull.
%% For Review pdf (With double line spacing)
%\documentclass[times,twocolumn,review]{elsarticle}
%% For abstracts longer than one page.
%\documentclass[times,twocolumn,review,longtitle]{elsarticle}
%% For Review pdf without preprint line
%\documentclass[times,twocolumn,review,nopreprintline]{elsarticle}
%% Final pdf
\documentclass[times,twocolumn,final]{elsarticle}
%%
%\documentclass[times,twocolumn,final,longtitle]{elsarticle}
%%


%% Stylefile to load MEDIMA template
\usepackage{medima}
\usepackage{framed,multirow}

%% The amssymb package provides various useful mathematical symbols
\usepackage{amssymb}
\usepackage{latexsym}
\usepackage{booktabs}
\usepackage{amsfonts}
\usepackage{amsmath}
\DeclareMathOperator*{\argmax}{arg\,max}
\DeclareMathOperator*{\argmin}{arg\,min}
\newcommand{\tabhead}{\textbf}
% Following three lines are needed for this document.
% If you are not loading colors or url, then these are
% not required.
\usepackage{url}
\usepackage{xcolor}

\usepackage{hyperref}

\definecolor{newcolor}{rgb}{.8,.349,.1}

\journal{Medical Image Analysis}

\begin{document}

\verso{M\'elanie Lubrano di Scandalea \textit{et~al.}}

\begin{frontmatter}

\title{Automatic grading of cervical biopsies by combining full and self-supervision}%
%\tnotetext[tnote1]{This is an example for title footnote coding.}

\author[1,2]{M\'elanie \snm{Lubrano di Scandalea}\corref{cor1}}
\cortext[cor1]{Corresponding author: 
  e-mail.: melanie.lubrano@keeneye.ai} 
\author[2,3,4]{Tristan \snm{Lazard}}
\author[1]{Guillaume \snm{Balezo}}
%% Third author's email
%\ead{author3@author.com}
\author[1]{Sylvain \snm{Berlemont}}
\author[2,3,4]{Thomas \snm{Walter}}
\address[1]{KEEN EYE, 74 Rue du Faubourg Saint Antoine, Paris, 75012, France}
\address[2]{Centre for Computational Biology (CBIO), MINES ParisTech, PSL University, 60 Boulevard Saint Michel, 75272 Paris Cedex 06, France}
\address[3]{Institut Curie, 75248 Paris Cedex, France}
\address[4]{INSERM, U900, 75248 Paris Cedex, France}

\received{}
\finalform{}
\accepted{}
\availableonline{}
\communicated{}


\begin{abstract}
%%%
In computational pathology, the application of Deep Learning to the analysis of Whole Slide Images (WSI) has provided results of unprecedented quality. Due to their enormous size, WSIs have to be split into small images (tiles) which are first encoded and whose representations are then agglomerated in order to solve prediction tasks, such as prognosis or treatment response. The choice of the encoding strategy plays a key role in such algorithms. Current approaches include the use of encodings trained on unrelated data sources, full supervision or self-supervision. In particular, self-supervised learning (SSL) offers a great opportunity to exploit all the unlabelled data available. However, it often requires large computational resources and can be challenging to train. On the other end of the spectrum, fully-supervised methods make use of valuable prior knowledge about the data but involve a costly amount of expert time. \\
This paper proposes a framework to reconcile SSL and full supervision and measures the trade-off between long SSL training and annotation effort, showing that a combination of both has the potential to substantially increase performance. On a recently organized challenge on grading Cervical Biopsies, we show that our mixed supervision scheme reaches high performance (weighted accuracy (WA): 0.945), outperforming both SSL (WA: 0.927) and transfer learning from ImageNet (WA: 0.877). We further provide insights and guidelines to train a clinically impactful classifier with a limited expert and/or computational workload budget. We expect that the combination of full and self-supervision is an interesting strategy for many tasks in computational pathology and will be widely adopted by the field.
%%%%
\end{abstract}

\begin{keyword}
%% MSC codes here, in the form: \MSC code \sep code
%% or \MSC[2008] code \sep code (2000 is the default)
%\MSC 41A05\sep 41A10\sep 65D05\sep 65D17
%% Keywords
\KWD mixed supervision\sep histopathology \sep whole-slide classification \sep self-supervised learning
\end{keyword}

\end{frontmatter}

%\linenumbers

%% main text


\section{Introduction}

Recent advances in slide digitization have led to increased  interest in Artificial Intelligence (AI) applications for histopathology. The development of AI models could help reduce pathologists’ workloads, limit subjectivity and help contributing to medical discoveries. Deep learning models can now match pathologist performance for many tasks: diagnostic, detection of mitoses \citep{veta_assessment_2015}, prediction of gene mutations \citep{coudray_classification_2018, kather_pan-cancer_2020} or genetic signatures \citep{kather_pan-cancer_2020, diao_human-interpretable_2021,lazard_deep_2021}, cancer subtyping \citep{coudray_classification_2018} and more. \\
One of the applications, automated diagnosis from Whole Slide Images (WSIs), induces two main challenges: first, WSIs are very high-resolution and, because of memory constraint, cannot be fed directly into traditional neural networks. Second, expert annotations are laborious to attain, costly and prone to subjectivity. The most popular methods today rely on Multiple Instance Learning (MIL), which frames the problem as a bag classification task. WSIs are split into small workable images (tiles), which are processed separately. Features from each of the individual tiles are extracted and then aggregated to classify the WSI. \\
The extraction of these tiles’ specific representation is crucial to the downstream WSI classification task. One common approach consists of initializing the feature extractor with pre-trained weights on ImageNet, a natural image dataset. This technique allows one to extract generic features that are powerful, but that do not lie within the histopathological domain. Different strategies have been developed to extract these tile encodings taking advantage of the available data and their respective level of supervision. \\
A first strategy aims to learn tile features with full supervision \citep{ehteshami_bejnordi_diagnostic_2017}. To create a supervised dataset, one or several experts manually review tiles and sort them into meaningful classes, preferably related to the downstream task of classifying the WSIs or equivalently perform the semantic segmentation of the WSI. Even though experts' annotations can bring powerful prior knowledge to the model, this technique often requires large quantities of annotations. \\
A second strategy consists of learning tile representations through self-supervision. It leverages the unannotated data by training a convolutional neural network on a pretext task. It has proven its efficacy \citep{saillard_identification_2021, lu_data-efficient_2021} and even its superiority over the fully supervised scheme \citep{dehaene_self-supervision_2020}. However, this approach has a non-negligible computational cost, as training necessitates around 1000 hours of computation on a standard GPU \citep{dehaene_self-supervision_2020}. Moreover, it is not guaranteed that the obtained encodings are most relevant for the prediction task we are trying to solve. \\
Techniques from both sides of the supervision spectrum have proven to bring important benefits for relevant feature extraction. Combining them could allow us to benefit from the best of both worlds. \\
In this work, in addition to proposing a joint-optimization process mixing self, full and weak supervision (Figure \ref{fig:pipeline_summary}), we measure the trade-off in performance between the number of annotations and the computational cost of training a self-supervised model. We thus provide guidelines to train a clinically impactful classifier with a limited budget in expert and/or computational workload. \\

\begin{figure*}[!t]
\centering
\includegraphics[scale=.35]{figures/pipeline_summary.pdf}
\caption{\textbf{Mixed Supervision Process}: \textbf{a)}A self-supervised model (SimCLR) is trained on unlabelled tiles extracted from the slides. Feature extractor and contrastive layer weights are transferred to the joint-optimization architecture \textbf{b)} Joint-optimization model is trained on the labeled tiles of the dataset. The feature extractor weights are transferred to the WS classification model. \textbf{c)} WS classification model is trained on the 1015 whole slide images.}
\label{fig:pipeline_summary}
\end{figure*}



\section{Related Work}

\paragraph{Mixed Supervision}

Annotations of medical data are often scarce, and one might want to take advantage of all that is available. For instance, whole slide images are often associated with one global label (weak supervision), they can contain millions of unlabelled tiles (no supervision), but, as a pathologist reviews the slides and performs a diagnostic, it is almost effortless for them to mark the region of interest that signs the corresponding diagnostic (strong supervision). AI applications have usually been dichotomized between supervised and unsupervised methods, spoiling the potential of combining several types of annotations. For this reason, mixing supervision for medical image analysis  has gained interest in recent years \citep{huang_rectifying_2020, li_thoracic_2018, li_hybrid_2021}.
For instance, in \citep{mlynarski_deep_2019} the author showed that combining global labels and local annotations by training in a multi-task setting, the capacities of the model to segment brain tumors on magnetic resonance images were improved.  \\
In \citep{tourniaire_attention-based_2021}, the author introduced a mixed supervision framework for WSI classification powering the CLAM \citep{lu_data-efficient_2021} architecture. They introduced supervision in the CLAM’s clustering task: instead of using pseudo-labels derived from attention weights for each tile, the annotated tiles were used.  In this work, the fully supervised information is injected at the WSI classification step. A mixture of WSI classification loss (weak supervision) and an instance-level classification loss (full-supervision) is optimized while, in contrast, we propose to optimize a combination of a contrastive loss (self-supervision) and an instance-level classification loss (full-supervision). \\
Their approach leads to an improvement of the WSI classification scores and disease localization with the attention score. It is, however, limited by the number of parameters that are concerned with this mixed supervision regime. While our method allows the fine-tuning of the whole feature extractor, the latter approach only fine-tunes a single linear layer downstream the feature extraction. Finally, the annotations used in \citep{tourniaire_attention-based_2021} are exhaustive: for the annotated slides, all the key regions are annotated pixel-wise. The resulting tile-level dataset is therefore bigger than the fully supervised dataset available for the TissueNet challenge.

\section{Materials and Method}

\subsection{Dataset and Problem Setting}

The Tissue Net Challenge \citep{drivendata_tissuenet_nodate} organized in 2020, the \textit{Soci\'et\'e Française de Pathologie (SFP)} and the \textit{Health Data Hub} aimed at developing methods to automatically grade lesions of the uterine cervix in four classes according to their severity. 
​​The training dataset for the challenge was made up of biopsy samples from female uterine cervix, focusing on epithelial lesions (Figure \ref{fig:uterine_cervix}). These lesions are often benign but can also be qualified as low grade or high grade depending on the risk of invasion of the underlying conjunctive tissue and evolution into carcinomas. The grade of the lesions depends on the proportion of squamous epithelium affected by dysplastic symptoms. The primary dysplastic symptom corresponds to the thickening of the basal layer. Low-grade squamous intraepithelial lesions (LSIL) are defined as having a dysplastic criteria involving less than one third of the thickness of the epithelium. High-grade squamous intraepithelial lesions (HSIL) indicate a greater proportion of the epithelium composed of undifferentiated basal cells with abnormalities. Carcinoma is diagnosed when abnormal epithelial cells invade the underlying conjunctive tissue. The class of a WSI was determined by the highest lesion's grade present on it.

\begin{figure}[!b]
\centering
\includegraphics[width=0.5\textwidth]{figures/cervix_dysplasia.jpg}
\caption{\textbf{Illustration of Uterine cervix dysplasia} - \citep{national_cancer_institute_definition_2011} }
\label{fig:uterine_cervix}
\end{figure}




\subsection{Fully supervised dataset}

5926 annotated Regions of Interest (ROIs) of fixed size 300x300 micrometers were provided. Each ROI had roughly the same size as a tile at 10x magnification and were labeled by the severity of the lesion it contained: “Normal” (0) if tissue was normal, (1) LSIL or (2) HSIL if it presented precancerous lesions that could have malignant potential and (3) invasive squamous carcinoma (Table \ref{tab:data-summary}).


\begin{table}[h]
\centering
\resizebox{0.5\textwidth}{!}{
\begin{tabular}{|c|cc|}
\hline
\textbf{Classes} & \textbf{Number of Slides} & \textbf{Number of Tiles} \\ \hline
\textbf{0 (Normal)}      & 270                       & 1923                     \\
\textbf{1 (Low Grade)}       & 288                       & 1405                     \\
\textbf{2 (High Grade)}       & 238                       & 1368                     \\
\textbf{3 (Carcinoma)}       & 219                       & 1230                     \\ \hline
\textbf{Total}   & 1015                      & 5926                     \\ \hline
\end{tabular}}
\caption{Dataset Summary}
\label{tab:data-summary}
\end{table}

\subsection{Weakly supervised dataset}
The dataset was composed of 1015 WSIs acquired from 20 different centers in France at an average resolution of 0.234 +/- 0.0086 mpp (40X). The slide resolution varied slightly due to the multicentric provenance of the data. The class of the WSI corresponded to the class of the most severe lesions it contained (grade from 0 to 3 also). All the native WS image formats were converted to pyramidal TIFF (Tagged Image File Format).
Both the WSI-level and tile-level labels have been attributed by a consortium of expert pathologists (Table \ref{tab:data-summary}).

\subsection{Misclassification Costs}

Misclassification errors do not lead to equally serious consequences (i.e predicting a benign lesion if it is cancerous is more serious than predicting a LSIL instead of a HSIL). Accordingly, a panel of pathologists established a grading of each of these errors i.e they attributed to each pair of possible outcome $(i,j) \in \{0, 1, 2, 3\}^{2}$ a severity score 
$0  \leqslant C_{i, j}  \leqslant 1$ (Table \ref{tab:error_table})  \\
The metric used in the challenge to evaluate and rank the submissions is computed from the average of these misclassification costs.\\
More precisely, if we name $P(S)$ the prediction of a slide $S$ labelled $l(S)$, the challenge metric $M_{WA}$ is: \\

\begin{equation}
	M_{WA} = \frac{1}{N} \sum_{S} (1-C_{l(S), P(S)})
\end{equation}
with $N$ the number of samples.\\
The problem is thus framed as a cost-sensitive classification problem, and, to our knowledge, all the winning solutions took awareness of this cost in their training procedure.

\begin{table*}[h]
\centering
\resizebox{0.8\textwidth}{!}{\begin{tabular}{l l l l l}
\toprule
\tabhead{Ground Truth} & \tabhead{Benign (pred)} & \tabhead{Low-grade (pred)} & \tabhead{High-grade (pred)} & \tabhead{Carcinoma (pred)} \\
\midrule
Benign & 0.0 & 0.1 & 0.7 & 1.0\\
Low-grade & 0.1 & 0.0 & 0.3 & 0.7\\
High-grade & 0.7 & 0.3 & 0.0 & 0.3\\
Carcinoma & 1.0 & 0.7 & 0.3 & 0.0\\
\bottomrule\\
\end{tabular}}
\caption{Weighted Accuracy Error Table - Error table to ponderate misclassification according to their gap with the ground truth
}
\label{tab:error_table}
\end{table*}


\section{Proposed Architecture}

\subsection{Multiple Instance Learning and Attention}

In Multiple Instance Learning, we are given sets of samples $B_k = \{x_i | i = 1 \ldots N_k\}$, also called bags. The annotation $y_k$ we are given refers only to the bags and not the individual samples. We assume however, that such tile-level labels exist in principle, but that we just do not have access to them. 

The strategy is to first map each tile $x_i$ to its encoding $z_i$, which is then mapped to a scalar value $a_i$, often referred to as attention score. The tile representations $z_i$ and attention scores $a_i$ are then agglomerated to build the slide representation $s_k$ which is then further processed by a neural network. The agglomeration can be based on tile selection \citep{campanella_clinical-grade_2019, courtiol_classification_2020}, or on an attention mechanism \citep{ilse_attention-based_2018}, which is today the most widely used strategy. 

\subsection{Self-supervised learning}

Self-supervised learning provides a framework to train neural networks without human supervision. The main goal of self-supervised learning is to learn to extract efficient features with inputs and labels derived from the data itself using a pretext task. Many self-supervised approaches are based on contrastive learning in the feature space. SimCLR, a simple framework relying on data augmentation was introduced in \citep{chen_simple_2020}. Powerful feature representations are learned by maximizing agreement between differently augmented views of the same data point via a contrastive loss applied in the feature space.

An image is transformed through random data augmentations into two new images. They are then embedded using the feature extractor. The two features vectors ($z_i$ and $z_j$) are mapped with a projection head (dense layers) to obtain final vectors $h_i$ and $h_j$ . The feature extractor and projection head are trained to maximize agreement using the contrastive loss. Positive pairs consist of the two augmented views of the same image, the other $2( n - 1 )$ views play the role of negative samples.
The loss function (NT-Xent) for a positive pair $(i,j)$ is defined as:

\begin{equation}
\mathcal{L}_{SSL} = -log \frac{exp(sim(h_i, h_j) / \tau}{\sum^{2n}_{k=1} \mathbf{1}_{k \neq i} exp(sim(h_i, h_j) \ \tau } 
\end{equation}
\\

Where $sim( u, v ) = \frac{u^{T} v}{|| u ||\cdot|| v ||}$ , the cosine similarity, $\mathbf{1}_{k \neq i \in (0, 1) }$  determines if $k \neq i$ and $\tau$ is a parameter. After convergence, the projection head is discarded and the pretrained feature extractor can be used for subsequent tasks.

\subsection{Cost-sensitive training}
Instead of the traditional cross-entropy loss we used a cost-aware classification loss, the Smooth-One-Sided Regression Loss $\mathcal{L}_{SOSR}$. First introduced to train SVMs in \citep{tu_one-sided_2010}, this objective function was smoothed and adapted for backpropagation in deep networks in \citep{chung_cost-aware_2016}. When using this loss, the network is trained to predict the class-specific risk   rather than a posterior probability; the decision function chooses the class minimizing this risk.  \\
The SOSR loss is defined as follows:

\begin{equation}
\mathcal{L}_{SOSR} = \sum_i \sum_j ln(1 + exp(\mathbf{2}_{i,j} \cdot (\hat{c}_i - \mathcal{C}_{i,j})))
\end{equation}

With $\mathbf{2}_{i,j} = - \mathbf{1}_{i \neq j}  + \mathbf{1}_{i = j}$ , $\hat{c_i}$ the $i$-th coordinate of the network output and $\mathcal{C}$ the error table.

\subsection{Mixed Supervision}
To be tractable, training of attention-MIL architectures requires freezing the feature extractor weights. While SSL allows the feature extractor to build meaningful representations \citep{saillard_identification_2021, dehaene_self-supervision_2020}, they are not specialized to the actual classification problems we try to solve. Several studies have shown that such SSL models benefit from fine-tuning specific to the downstream task \citep{chen_simple_2020} \\
We therefore added a training step to leverage the tile-level annotation and fine-tune the self-supervised model.
However, as the final WSI classification task is not identical to the tile classification task, we suspect that fine-tuning solely on the tile classification task may over-specialize the feature extractor and thus sacrifice the generalizability of SSL (and for this reason ultimately also degrading the WSI classification performances).
To avoid this, we developed a training process that optimizes the self-supervised and tile-classification objectives jointly.

Two different heads, plugged before the final classification layer, are used to compute both loss functions $\mathcal{L}_{SSL}$  and $\mathcal{L}_{SOSR}$
The final objective $\mathcal{L}$ is then:

\begin{equation}
\mathcal{L} = \beta \mathcal{L}_{SSL} + (1- \beta) \mathcal{L}_{SOSR}
\label{eq:joint-eq}
\end{equation}

where $\beta$ is a hyperparameter that has to be tuned. Here, we found $\beta=0.3$ (see Supplementary).

\section{Understanding the feature extractor with Activation Maximization}
To further understand the features learned by the different pre-training policies (ImageNet, supervised, SSL and mixed), we used Activation Maximization (AM) to visualize extracted features  and provide an explicit  illustration of the specificity learned. \\
Methods to generate pseudo-images maximizing a feature activation have been introduced in \citep{erhan_visualizing_2009}. This technique consists in synthesizing the images that will maximize one feature activation. It is summarized as follow \citep{nguyen_understanding_2019}: \\
If we consider a trained classifier with set of parameters $\theta$ that map an input image $x \in \mathbb{R}^{h \times w \times c} $, ($h$ and $w$ are the height and width and $c$ the number of channels) to a probability distribution over the classes, we can formulate the following optimization problem:

\begin{equation}
x^{*} = \argmax_{x}(\sigma^{l}_{i}(\theta, x))
\end{equation}

where $\sigma^{l}_{i}(\theta, x)$ is the activation of the neuron i in a given layer l of the classifier. This formulation being a non-convex problem, local maximum can be found by gradient ascent, using the following update step:

\begin{equation}
x_{t+1} = x_{t} + \epsilon \frac{\partial \sigma^{l}_{i}(\theta, x)}{\partial x_t}
\end{equation}


The optimization process starts with a randomly initialized image. After a few steps, it generates an image which can help to understand what information is being captured by the feature. 
As we try to visualize meaningful representations of the features, some regularization steps are applied to the random noise input (random crop and rotations to generate more stable visualization, details can be found in Supplementary Materials). To generate filter visualization within  the HE space, we transformed the RGB random image to HE input thanks to color deconvolution \citep{ruifrok_quantification_2001}. This preprocessing allowed to generate images with histology-like colors when converted back to the RGB space. \\
To select the most meaningful features for each class, we trained a Lasso classifier without bias to classify the extracted feature vectors into the four classes of the dataset for the four pre-training policies. The feature vectors for each tile were first normalized and divided element-wise by the vector of features’ standard deviation across all the tiles.  The L1 regularization factor  $\lambda$ was set to 0.01. Details about Lasso training can be found in Supplementary Materials. Contribution scores for each feature were therefore derived from the weights of the Lasso linear classifier: negative weights were removed and remaining positive weights were divided by their sum to obtain contribution scores $[0, 1]$. By filtering out the negative weights, the contribution score corresponds to the proportion of attribution among the features positively correlated to a class, and allows to select feature capturing semantic information related to the class, leaving out  those containing information for other classes.



\section{Experimental Setting}

\subsection{WSI preprocessing}

Preprocessing on a downsampled version of the WSIs was applied to select only tissue area and non-overlapping tiles of 224x224 pixels were extracted at a resolution of 1 mpp. (Details in Supplementary Materials)

\subsection{Data splits for cross-validation}
To measure the performances of our models we performed 3-fold cross-validation for all our training settings. Because the annotated tiles used in our joint-optimization step were directly extracted from the slides themselves, we carefully split the tiles such that tiles in different folds were guaranteed to originate from different slides. The split divided the slides and tiles into a training set, a validation set and a test set. \\ 
All subsequent performance results are then reported as the average and standard deviation of the performance results on each of these 3 test folds.

\subsection{Feature extractor pre-training}
The feature extractor is initialized with pre-trained weights obtained with three distinct supervision policies: fully supervised, self-supervised or a mix of supervision. These three policies rely on the fine-tuning of a DenseNet121 \citep{huang_densely_2016}, pretrained on ImageNet. The fully-supervised archicture is fine-tuned solely on the tile classification task. The SSL architecture is derived from SimCLR framework and is trained on an unlabeled dataset of 1 million tiles extracted from the slides. Finally for the mixed-supervised architecture, a supervised branch is added to the previous SSL network and trained using the mixed objective function (see Fig. \ref{fig:pipeline_summary} and Eq. \ref{eq:joint-eq}) on the fully supervised dataset. Technical details of these three training settings are available in the supplementary material.


\subsection{Whole Slide Classification}
After tiling the slides, the frozen feature extractor (DenseNet121) was applied to extract meaningful representations from the tiles. This feature extractor was initialized sequentially with the pre-trained weights mentioned above and generated as many sets of features. These bags of features were then used to train the Attention-MIL model with SOSR loss applied slide-wise. (Supplementary Materials).

\subsection{Feature Visualization}
To select the most relevant features, we trained an unbiased linear model on the feature vectors extracted from the annotated tiles. The feature vectors were standardised. The weights of the linear model were used to determine which features were the most impactful for each class. Feature visualizations were generated for the selected features and for each set of pre-trained weights. We extracted the tiles expressing the most of these features by selecting the feature vectors with the higher activation for the concerned feature. Implementation details are provided in Supplementary Materials. 



\section{Results}

\subsection{Self-supervised fine-tuning}

We saved the checkpoints of the self-supervised feature extraction model at each epoch of training, allowing us to investigate the amount of time needed to reach good WSI classification performances. We computed the embeddings of the whole dataset with each of the checkpoints and trained a WSI classifier from them. Figure \ref{fig:ssl_epochs} reports the performances of WSI classification models for each of these checkpoints.
SSL training led to a higher Weighted Accuracy than using ImageNet weights after 3 epochs and resulted in a gain of  +4.8\% after 100 epochs.
Interestingly, as little as 6 epochs of training are enough to gain 4\% of Weighted Accuracy: a significant boost in performance is possible with 50 GPU-hours of training.
We then observe a small increase in performance until the 100th epochs.


\begin{figure}[h]
\centering
\includegraphics[width=0.5\textwidth]{figures/epochs_ssl.jpeg}
\caption{\textbf{Weighted Accuracy evolution} - Weighted Accuracy evolution on WS classification task with respect to the number of epochs of SSL training}
\label{fig:ssl_epochs}
\end{figure}


\subsection{Pre-training policy comparison}
To compare the weights obtained with the various supervision levels, we ran a 3-fold cross-validation on the WS classification task and  summarized the results in table \ref{tab:pretraining}. 
The results indicate that SSL training substantially improves the WSI classification performance. In contrast, we see that initializing the feature extractor with fully-supervised weights gives an equivalent or poorer performance than any other initialization. SSL training allows us to extract rich features that are generic, yet still relevant to the dataset  (unlike ImageNet). On the other hand, fully supervised features are probably too specific and seem to not represent the full diversity of the image data. The joint-optimization process manages to balance out generic and specialized features without neutralizing them: mixing the supervision levels brings significant improvements (+2\%) to the performance, leading to a Weighted Accuracy of 0.945. \\
We additionally compared the benefits introduced by the cost-sensitive loss [REF equation SORS] with the cross-entropy loss. Our results show that with ImageNet weights the SORS loss improves the Weighted Accuracy by 1\% and the accuracy by 3\%. \\
In conclusion, the combination of the SSL pre-trained model, its fully supervised fine-tuning, and the cost-sensitive loss leads to a notable improvement of 8 Weighted Accuracy points over the baseline MIL-imagenet model.


\begin{table}[!t]
\centering
\begin{tabular}{ccc}
\hline
                    & \textbf{Accuracy}        & \textbf{SFP\_metric}     \\ \hline
imagenet+ce         & 0.758 +/- 0.034          & 0.865 +/- 0.023          \\
imagenet+sors       & 0.787 +/- 0.032          & 0.877 +/- 0.029          \\
supervised+sors     & 0.772 +/- 0.055          & 0.874 +/- 0.027          \\
ssl+sors            & 0.803 +/- 0.016          & 0.925 +/- 0.006          \\
\textbf{mixed+sors} & \textbf{0.845 +/- 0.028} & \textbf{0.945 +/- 0.005} \\ \hline
\end{tabular}

\caption{\textbf{Pre-training policies} - Performances summary}
\label{tab:pretraining}
\end{table}

\subsection{Number of annotations vs Number of epochs}

We have seen that both SSL and supervised training bring an improvement in the WSI classification task. To further investigate the relationship between these two supervision regimes, we trained models with only some of the fully supervised annotations (15, 65, 100\%) on top of intermediate SSL checkpoints.
Results are reported in table \ref{tab:annots_epochs}. \\
It appears that without SSL training (or with too few epochs of training), the supervised fine-tuning does not bring additional improvement in WSI classification. This is in line with the work of Chen et al. (Chen, Kornblith, Swersky, et al. 2020) that showed that an SSL model is up to 10x more label efficient than a supervised one. \\
However, for the 100-epoch checkpoints, we observe an improvement of 2 points of the Weighted Accuracy when using 100\% of the tile annotations.
Moreover, fine-tuning the models by mixed supervision with too few annotations (15\%) leads to a slight drop in WSI classification performances.
Finally, we see  a diminution of the standard deviations across splits for the different pre-training policies, showing better stability for longer SSL training and more annotations. \\

We draw different conclusions from these observations:
\begin{itemize}
\item In this context, it is always better to pre-train the feature extractor with SSL rather than only invest in annotations.
\item The supervised fine-tuning needs enough annotations to bring an improvement to the WSI classification task. We can note however that even when considering the 100\% annotation settings, the supervised dataset ( approx. 5000 images) is still rather small in comparison to traditional image datasets.
\item A full SSL training is mandatory to leverage this small amount of supervised data. 
\end{itemize}

\begin{table*}[h]
\centering
\resizebox{0.7\textwidth}{!}{
\begin{tabular}{|c|c|c|c|c|}
\hline
\textbf{} &
  \textbf{0 Annot.} &
  \textbf{\begin{tabular}[c]{@{}c@{}}$\sim$1 Annot. / slide\\ (1015 tiles)\end{tabular}} &
  \textbf{\begin{tabular}[c]{@{}c@{}}$\sim$4 Annot. / slide\\ (3901 tiles)\end{tabular}} &
  \textbf{\begin{tabular}[c]{@{}c@{}}$\sim$6 Annot. / slide\\ (5926 tiles)\end{tabular}} \\ \hline
\textbf{ImageNet (no SSL)} & 0,877 +/- 0.029 & 0.872 +/- 0.024 & 0.872 +/- 0.023 & 0,874 +/-0.027           \\ \hline
\textbf{SSL-epoch10}       & 0,912 +/- 0.019 & 0,907+/- 0.024  & 0,903 +/- 0.029 & 0,916 +/- 0.019          \\ \hline
\textbf{SSL-epoch50}       & 0,915 +/- 0.014 & 0,913 +/- 0.024 & 0,916 +/- 0.014 & 0,914 +/- 0.022          \\ \hline
\textbf{SLL-epoch100}      & 0,925 +/- 0.006 & 0,916 +/- 0.010 & 0,921 +/- 0.010 & \textbf{0,945 +/- 0.005} \\ \hline
\end{tabular}}
\caption{\textbf{Relationship between self-supervision and full-supervision} - Study on the performance improvement on WS classification for different proportion of labelled data versus different training time of SSL}
\label{tab:annots_epochs}
\end{table*}


\subsection{Features Visualisations}

\begin{figure*}[!t]
\centering
\includegraphics[width=1\textwidth]{figures/filters_class0.pdf}
\caption{\textbf{Feature Visualization} - Top Features for class "Normal" (0) and associated tiles}
\label{fig:top_features_0}
\end{figure*}

\begin{figure*}[h]
\centering
\includegraphics[width=0.8\textwidth]{figures/filters_classes_comparison.pdf}
\caption{\textbf{Feature comparison per class} - The top row displays the top filter for the Mixed Supervised model for each class. The bottom row displays the tile expressing the feature the most}
\label{fig:features_comp}
\end{figure*}

\begin{figure*}[!t]
\centering
\includegraphics[width=1\textwidth]{figures/feature_redundancy_top1to5_class3.pdf}
\caption{\textbf{Feature Diversity for the class "Carcinoma" (3) (top 5 features)} - Class "Normal" (0) and top 10 features in Supplementary Materials}
\label{fig:features_diversity}
\end{figure*}


We generated the pseudo-images of the most important features for each class and each pre-training policy and extracted the related tiles. The Figure \ref{fig:top_features_0} displays the most important features along with the tiles activating each feature the most for the class “Normal” (0). Although interpretation of such pseudo images must be treated carefully, we notice that the features obtained with SSL, supervised and mixed training are indubitably more specialized to histological data than those obtained with ImageNet. Some histological patterns, such as nuclei, squamous cells or basal layers are clearly identifiable in the generated images. The extracted tiles are strongly correlated with class-specific biomarkers. Feature e highlights well organized basal layers, whereas features \textbf{c} and d highlight clouds of regular nuclei. Feature \textbf{g} and \textbf{h} are characteristic of squamous cells (polygonal shapes, stratified organization lying on a straight basal layer). While features from ImageNet (\textbf{a}, \textbf{b}), SSL (\textbf{c}, \textbf{d})  and the supervised model (\textbf{g}, \textbf{h}) focus on the upper half of the cervix epithelium, it appears that features from the mixed supervision model (\textbf{e}, \textbf{f}) are focusing on the lower half which is known to be the relevant region for discrimination  between class Normal (0) and Low Grade (1) (abnormal cells are constricted to the lower third of the epithelium), suggesting that mixed supervision highlights pathologically relevant patterns to a larger extent than the other regimes \citep{who_colposcopy_2020}. \\
In Figure \ref{fig:features_comp} we can further identify class-related biomarkers for dysplasia and carcinoma grade. Tiles with visible koilocytes (cells with a white halo around the nucleus) have been extracted from the top features for Low Grade class. Koilocytes are symptomatic of infection by Human Papillomavirus  (almost always responsible for precancerous lesions in the cervix,  \citep{who_colposcopy_2020}). High Grade (2) generated image represents disorganised cells with large nuclei and no polarity. For the class “Carcinoma” (3), we observe cluster patterns of large nuclei with irregular aspects (similar to prominent nucleoli), separated by fibrous texture, that can be identified as stroma patterns. All these criteria have been identified in \citep{who_colposcopy_2020} as biomarker patterns for precancerous lesions of the cervix. \\
In Figure \ref{fig:features_diversity}, we observe that features extracted from ImageNet and SSL models are diverse, in particular, features extracted from SSL reflect rich tissue phenotypes which correlates to their generic capacities of image representations. On the other hand, features extracted with supervised and mixed methods are more redundant. 
We additionally observe in Figure \ref{fig:features_diversity} that feature visualisation from the mixed model picture realistic histopathological patterns specific to the class. Visualisation for other classes are available in Supplementary Materials.



\section{Discussion}

In pathology , expert annotations are usually hard to obtain. However, we are often in a situation where a small quantity of labeled annotation exists but not in sufficient quantities to support  fully supervised techniques. Yet, even in small quantities, expert annotations carry meaningful information that one could use to enforce biological context to deep learning models and make sure that networks learn appropriate patterns. On the other hand, self-supervised methods have proven their efficacy to extract generic features in the histopathological domain and their usefulness for downstream supervision tasks, even in the absence of massive ground truth data. Methods capable of reconciling self-supervision with strong supervision can therefore be useful and open the door to better performances. \\

In this paper, we presented a way to inject the fine-grained tile level information by fine-tuning the feature extractor with a joint optimization process. This process allowed to mix self-supervised learning features with tile classification ones and helped the downstream WSI classification task.  \\

We applied our method to the TissueNet Challenge, a challenge for the automatic grading of cervix cancer, that provided annotations at the slide and tile level, thus representing an appropriate use case to validate our method of mixed supervision. We also propose in this study insights and guidelines for the training of a WSI classifier in the presence of tile annotations. \\

First, we showed that SSL is always beneficial to our downstream WSI classification tasks. Fine-tuning pre-trained weights with SSL for only 50 hours brings a 4\% improvement over WSI classification weighted accuracy, and near to 5\% when fine-tuning for longer (100 epochs). \\

Second, a small set of annotated tiles can bring benefit to the WSI classification task, up to 2\% of weighted accuracy for a supervised dataset of around 5000 images. \\
Such a set of tiles can be obtained easily by asking the pathologist to select a few ROIs that guided his decision while labeling the WSIs, which can be achieved without a strong time commitment.
However this boost in performance can be reached only if the feature extractor is pre-trained with SSL, and for sufficiently long: SSL unlocks the supervised fine-tuning benefits.\\

To further understand the differences between the range of supervision used to extract tile features, we conducted qualitative analysis on features visualizations by activation maximization and observed that features obtained from SSL, supervised or mixed trainings were more relevant for histological tasks and that class-discriminative patterns were indeed identified by the model. We also observed that supervised training on the tiles alone led to much less diversity in the features extracted by the model than the ones obtained with SSL.\\

The scope of this study contains by design three limitations.
First, SSL models were trained by fine-tuning already pre-trained weights on imagenet. This may explain the rapid convergence and boost in performance observed; However it may also underestimate this boost if the SSL models were trained from scratch. We did not compare SSL trained from scratch and fine-tuned SSL, and left it to future work. \\

Second, all the conclusions reached are conditioned by the fact that we do not fine-tune the feature extractor network during the WSI classification training. Keeping these weights frozen, and even pre-computing the tiles representations brings a large computational benefit (both in memory and speed of computations), but prevents the feature extractor from specializing during the WSI classification training. \\

Third, the tendency observed in table \ref{tab:annots_epochs} of better performances correlated with larger numbers of annotations is modest and would require more annotations to validate it. \\

Finally, our method can be improved in several ways. First, SimCLR, was a pioneer method in self-supervised learning architecture and has proven to be efficient but it suffers from high performance drop when decreasing the batch size \citep{chen_simple_2020}. Other SSL models have been developed to alleviate this limitation. MoCo \citep{he_momentum_2020} actually propose a momentum mechanism allowing optimal performances even without large batch size and therefore, numerous available parallel GPUs. Other models like VICReg \citep{bardes_vicreg_2021} proposed  techniques to maximize the variance between the features and therefore limit their redundancies. It will be interesting to benchmark these SSL variants with respect to their impact on WSI classification accuracy and feature interpretability. \\

To conclude, we present a method that provides an interesting alternative to using full supervision, pre-training on unrelated data sets or self-supervision. We convincingly show that the learned feature representations are both leading to higher performance and providing more intermediate features that are more adapted to the problem and point to relevant cell and tissue phenotypes. We expect that the mixed supervision will be adopted by the field and lead to better models. 










\section*{Acknowledgments}
The authors thank Ya\"elle Bellahsen-Harrar for providing insightful trainings about the medical problem. ML was supported by a CIFRE PhD fellowship founded by KEEN EYE, Paris, France and ANRT. TL was supported by a Q-Life PhD fellowship (Q-life ANR-17-CONV-0005). Furthermore, this work was supported by the French government under management of Agence Nationale de la Recherche as part of the "Investissements d'avenir" program, reference ANR-19-P3IA-0001 (PRAIRIE 3IA Institute).

%\section*{References}

%%Harvard
\bibliographystyle{model2-names.bst}\biboptions{authoryear}
\bibliography{mybibfile}



\end{document}

%%
